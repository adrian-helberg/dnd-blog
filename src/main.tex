%% Author: Adrian Helberg
%% 05.02.2019

\documentclass[11pt, openany]{article} % Preamble
\usepackage{amsmath} % Packages
\usepackage[ngerman]{babel}% Language
\usepackage{fontspec} % Font
%\usepackage[urf8]{inputenc} % Note: Not for LuaLaTeX-Compiler
\setmainfont[ItalicFont=Dominican Italic]{Dominican}
\usepackage[absolute]{textpos} % Positioning
\usepackage[percent]{overpic} % Grafiken
\usepackage{color} % Farben zB. Schriftfarbe
\usepackage[linktoc=all]{hyperref} % Anker Links
\usepackage{lettrine} % Let the first letter of a word be capital

% Anker Setup
\hypersetup{
colorlinks,
citecolor=black,
filecolor=black,
linkcolor=black,
urlcolor=black
}

% Title
\title{\fontsize{40pt}{42pt}\selectfont Dungeons \& Dragons \\
Rathians Hort
\author{Schriftf\"uhrer: Der Dungeonmaster}
\date{Letzte Federf\"uhrung: \today}
\begin{textblock}{4}(6,10)
\Huge Reisetagebuch
\end{textblock}
}

% Table of contents additions
\addto\captionsngerman{
\renewcommand{\contentsname}
{Schlachtplan}
}

% Force linebreaking on words that would be hyphenated
\tolerance=1
\emergencystretch=\maxdimen
\hyphenpenalty=10000
\hbadness=10000

% Content
\begin{document}

\maketitle
\newpage
\tableofcontents
\newpage

%%%%%%%%%%%%%%
% Einleitung %
%%%%%%%%%%%%%%
\section{Einleitung}

\lettrine{G}{ egr\"u"st} seist du werter Wanderer, abenteuerlicher Gesell!\\ Das Fabelhafte, das du hier lesen wirst, schreibt sich als Erfahrung einer Gruppe von wackeren Recken nieder, die einmal Abenteuerluft geschnuppert, auf eine Reise ins Unbekannte aufbrachen. Nimm nur all deinen Mut zusammen und folge ihnen durch Raum und Zeit; Werde Zeuge einer Menge Spa"s, Frust und Angstschwei"s!\\

Doch zun\"achst bedenke, dass die Geschichte nicht abgeschlossen oder vollst\"andig ist. Da unsere Rollenspielgruppe zur Zeit etwa 2-3 Stunden pro Woche in die Gestaltung der Abenteuer investieren kann, wird dieser Bericht stetig erweitert und weicht von den Geschehnissen in unseren D\&D-Sessions ab. Des Weiteren ergibt sich mir als Story-Teller diesbez\"uglich die M\"oglichkeit die Gemeinschaft mit Informationen \"uber die Welt zu versorgen. \\

Die Welt, wie die Gruppe sie formt, erhebt keinen Anspruch auf Deckungsgleichheit mit dem Niedergeschriebenen aus den Dungeons \& Dragons B\"uchern. Eher habe ich eine mehr-oder-weniger konkrete Vorstellung meiner eigenen Welt und hoffe, dass der Leser den einen oder anderen Konsistenzfehler entschuldigt. F\"ur Verbesserungsvorschl\"age bin ich jederzeit offen!\\

Man wird obendrein meine Vorliebe f\"ur das \mbox{''\textit{Das Schwarze Auge}''} - Universum feststellen, da ich dessen Welt Aventurien und dessen G\"otterpantheon des Zw\"olfg\"otterglaubens adaptiert habe.\\

\begin{center}
    GENUG GEFASELT\\
    - Ein Zwerg raunt mich von der Seite an -\\
    DU BIST MIT DER N\"ACHSTEN RUNDE DRAN!\\
    - Ich sollte ihn nicht darauf aufmerksam machen, -\\
    - dass ein Brocken Kalbsleber in seinem Barte h\"angt... -\\
    - Ach was solls, wer will schon ewig Leben?! -
\end{center}

\newpage

\subsection{Lebenswei"sheiten}

\lettrine{M}{ an} f\"uhre sich folgende Zitate zu Gem\"ute. Diese Abschrift wird stetig\\ erweitert. Lesen auf eigene Gefahr!

\begin{quote}
    \textit{"Die Geschichten des Reisetagebuchs entspringen einer Reihe von D\&D-Sessions, sprich: Eine Aneinanderreihung von Bl\"odeleien, Unsinn und empirischer Feldforschung"}
    \begin{flushright}
        - Ein D\&D-Spieler
    \end{flushright}
\end{quote}

\begin{quote}
    \textit{"Wer im Gasthaus sitzt, sollte nicht mit Schweinen werfen!"}
    \begin{flushright}
        - Ein Zwerg
    \end{flushright}
\end{quote}

\begin{quote}
    Elfische Volksweisheit, stark vereinfacht: \textit{"Wer seine Artgenossen befriedigt, hat Selbstbefriedigung erlangt"}
    \begin{flushright}
        - Ein Elf
    \end{flushright}
\end{quote}

\begin{quote}
    Mitstreiter infiltriert Piratenschiff, schleicht sich an Deck und versteckt sich (kritisch verpatzter Wurf) an Deck.\\ Im Klartext: Dieser Taugenichts sitzt auf dem Vordeck mit einem leeren Fass auf dem Kopf und denkt er sei versteckt...\\ Das bleibt nat\"urlich nicht unbemerkt:\\
    Pirat: \textit{''WER BIST DU?!''}\\
    Jene verkrachte Existenz: \textit{''Ich bin der Neue!''}\\ - Man denke sich \"ubertriebenen Enthusiasmus -
    \begin{flushright}
        - Zum Charakter oder Spieler m\"ochte ich schweigen.\\
        Oder besser noch: Eine Schweigeminute einlegen
    \end{flushright}
\end{quote}

\begin{quote}
    \textit{''Unsere Gruppe ist am Arsch!''}\\
    \textit{''Warum? Was ist passiert?''}\\
    \textit{''Unser Ork hat einen Intelligenz-Trank getrunken und besteht nun darauf sich beim \"ortlichen Praioten eine schriftliche Vollmacht zur Ergreifung des Diebes zu holen... Da l\"auft er \"ubrigens''}
    \begin{flushright}
        - Wenn man als Meister mal kurz nicht hinh\"ort
    \end{flushright}
\end{quote}

\begin{quote}
     \textit{''Wenn man etwas von Elfen will, muss man ihnen ein profanes Opfer bringen; z.B. den Busch einer Dryade gie"sen!''}\\
     - Ein leises Kichern macht sich breit -\\
     \textit{''Nicht so wie DU denkst!''}\\
     \textit{''Ja, dann sind w\"ar sie n\"amlich ziemlich angepisst!''}
     \begin{flushright}
         - *Kopfsch\"uttel*
     \end{flushright}
\end{quote}

\begin{quote}
    \textit{''Der Magier hat sich wegteleportiert!''}\\
    \textit{''Und WIE soll er das gemacht haben?''}\\
    \textit{''Vermutlich durch den Limbus...''}\\
    - Kurze Pause -\\
    \textit{''Und wie lange f\"ahrt der Limbus?''}\\
    - Gel\"achter setzt ein -
    \begin{flushright}
        - "Der Neue"
    \end{flushright}
\end{quote}

\subsection{Zwergenfaustregeln}

\textit{''Wie habe ich diesen Blog/diese Geschichte zu lesen?''}\\
Hier einmal ein paar Dinge zu diesem Thema:
\begin{enumerate}
    \item[$\rightarrow$] Der Hintergrund zu den Charakteren stammt im gro"sen Teil von den Spielern
    \item[$\rightarrow$] Die Gruppe trifft sich einmal pro Woche in sog. ''Sessions'', um Abenteuer zu bestreiten, sich in der Diplomatie zu \"uben oder sich grundlos zu besaufen ... woraus diese Geschichte entsteht
    \item[$\rightarrow$] Dieses Dokument wird nur pseudo-regelm\"a"sig aktualisiert, allerdings mind. einmal pro Woche (wenn es die Zeit zul\"asst). Das ist zumindest der Plan
    \item[$\rightarrow$] Bei dem Text im Schlachtplan handelt es sich um Ankerlinks
\end{enumerate}

\newpage

%%%%%%%%%%%%%%%%%%%%%%%%
% Charaktervorstellung %
%%%%%%%%%%%%%%%%%%%%%%%%
\section{Dramatis Personae}

''\textit{Helden? Wer hat hier etwas von Helden erz\"ahlt? Nein die folgenden Charaktere sind keine Helden. Eher sowas wie hineingestolperte Personen, die hoffen irgendwie klarzukommen}''\\
Da diese Illusion jetzt auch vom Tisch ist: Hier die Leutchen, die der Leser kennenlernen sollte.

\subsection{Heagan}

\lettrine{H}{ eagan} ist ein junger, mittelgro"ser Mann von 2 Jahrzehnten mit azurblauen Augen und kupfernem Haar.
Als junges Kind wurde er im Peraine-Tempel aufgenommen, da seine Eltern sich nicht um ihn k\"ummern konnten.

\begin{figure}[h]
    \begin{overpic}[width=\textwidth, height=0.12\textheight]{../img/parchment_light.jpg}
        \put (2,14) {Peraine ist die G\"ottin des Ackerbaus und der Heilkunst.}
        \put (2,10) {Sie wird von der \"uberwiegenden Mehrheit der Bev\"olkerung verehrt, da diese}
        \put (2,6)  {l\"andlich strukturiert ist. Das Wappen Peraines ist eine goldene \"Ahre auf}
        \put (2,2)  {gr\"unem Grund. Eine als peraineheilig geltende Frucht ist der Goldapfel.}
    \end{overpic}
\end{figure}

Dort wurde er von den Priestern vieles gelehrt; Vor allem den Umgang mit Pflanzen und das Versorgen der Kranken. Die meiste Zeit sa"s er vertieft in die Schriften des Tempels um dessen Wissen in sich aufzunehmen. Wenn Menschen in den Tempel kamen und der Heilung bedurften, pflegte Heagan sie, bis sie gesund waren. Weiter hatte er weniger Kontakt mit den Menschen au"serhalb des Tempels. Das Wichtigste, das ihn die Priester lehrten war die Vermeidung von Konflikten und das Sch\"utzen jeglichen Lebens so gut er konnte.  Mithilfe dieses Wissens ist er in der Lage auch au"serhalb des Tempels seine eigenen Wege zu finden. Die Gesetze der G\"ottin sind ihm heilig und er h\"alt sich stets an diese; Das hat er sich geschworen, als er damals fortging.

\newpage

\subsection{Kerlia Sanae}

\lettrine{D}{ ie} freiheitsliebende Kerlia ist eine junge Hochelfin von mittlerem Wuchs und hellblauen Augen. Das rote Haar und die blasse Haut laden zum Tr\"aumen ein.

\begin{figure}[h]
    \begin{overpic}[width=\textwidth, height=0.18\textheight]{../img/parchment_light.jpg}
        \put (2,22) {Hochelfen sind magiebegabte Elfen, die im Allgemeinen viel Wert auf Gr\"o"se,}
        \put (2,18) {Prunk und Kunstfertigkeit, Philosophie, Kunst, Musik, Architektur,}
        \put (2,14) {Rhetorik und andere intellektuelle und k\"unstlerische Bet\"atigungen legen.}
        \put (2,10) {F\"ur den Menschen wirken Hochelfen  unfassbar sch\"on, was wohl ihrer}
        \put (2,6) {makellosen Haut, der Weichheit ihrer Sprache und dem perfekten K\"orperbau}
        \put (2,2) {zuzusprechen ist.}
    \end{overpic}
\end{figure}

Doch der \"au"sere Schein kann tr\"ugen! Kerlia ist eine Piratin durch und durch. Sich selbst sieht sich als diejenige, die von den Leuten, die zu viel haben, nimmt und den weniger betuchten gibt. Nichtsdestotrotz k\"onnen diese, welche sich zu ihren Freunden und Mitstreitern z\"ahlen auf Kerlia verlassen und ihr vertrauen. In St\"adten und auf hoher See f\"uhlt sie sich zuhause und ihre gr\"o"ste Angst ist ein K\"afig.

\subsection{Koriel Mystan}

\lettrine{S}{ ieht} man einen m\"annlichen Drachengeboreren, der 2,20m gro"s ist, hellblaue Augen hat und durch seinen wei"sen Schuppenpanzer unter allen anderen heraussticht, hat man h\"ochst wahrscheinlich Koriel ins Auge gefasst.

\begin{figure}[h]
    \begin{overpic}[width=\textwidth, height=0.15\textheight]{../img/parchment_light.jpg}
        \put (2,18) {Die Drachengeborenen sind eine Vermischung aus Drachen und Humanoiden.}
        \put (2,14) {Sie gehen stolz durch eine Welt, die sie mit furchterregendem Unverst\"andnis}
        \put (2,10) {begr\"u"st. Viele Drachengeborenen stellen sich unter den Dienst echter Drachen}
        \put (2,6)  {und andere bilden Soldaten in gro"sen Kriegen oder folgen keiner besonderen}
        \put (2,2)  {Berufung.}
    \end{overpic}
\end{figure}

Koriel wurde in einer Enklave von Dragonborn vor 50 Wintern geboren. Er wuchs beh\"utet und in Frieden auf. Sein wei"ser Schuppenpanzer hat ihn damals schon zu entwas Besonderem gemacht. Sein Weg f\"uhrte ihn in das Milit\"ar des K\"onigs, wo er immer gern gesehen war und schon bald zum Kommandanten aufstieg. Eines Tages ergriff Koriel die M\"oglichkeit etwas Gro"ses aus seinem Namen zu machen.

\subsection{Rurik Eisenfaust}

\lettrine{W}{ er} am Praiostag einer ausschweifenden Messe beiwohnen m\"ochte, k\"onnte dort auf Rurik treffen. Ein Zwergen-Priester wie er im Buche steht; Und das tut dieser auch: Die ehrenwerte Liturgie der Paraioskirche.

\begin{figure}[h]
    \begin{overpic}[width=\textwidth, height=0.18\textheight]{../img/parchment_light.jpg}
        \put (2,22) {Praios ist der h\"ochste Gott des Pantheons der Zw\"olfg\"otter.}
        \put (2,18) {In Gestalt der Sonne soll er jeden Tag in unver\"anderlicher Ordnung seine Bahn}
        \put (2,14) {\"uber den Himmel ziehen. Die pr\"achtigsten und reichsten Tempel sind Praios}
        \put (2,10)  {geweiht, auch wenn man sie seltener auf dem Lande als in der Stadt findet}
        \put (2,6)  {Als Gott von Recht und Ordnung wird vielerorts in seinem Namen Gericht}
        \put (2,2)  {gehalten.}
    \end{overpic}
\end{figure}

Ein f\"ur Zwergen mittlerer Wuchs, ein Paar braune Augen und das rostrote Haar, das wie der pr\"achtige Bart, in langen Z\"opfen geflochten ist, zeichnen diesen Geweihten aus. Rurik vertritt seinen Glauben in allen Lebenslagen und hat somit immer ein offenes Ohr f\"ur die Probleme und Pr\"ufungen anderer Mitstreiter und somit kam es, dass er sich auf eine spirituelle Fahrt ins Ungewisse machte und dort gern als S\"aule des Glaubens fungiert.

\subsection{Thurmond}

\lettrine{T}{ hurmond} ist ein waschechter Druide! Ein von Gaia auserw\"ahlter Diener der Natur und der Elementargeister.

\begin{figure}[h]
    \begin{overpic}[width=\textwidth, height=0.28\textheight]{../img/parchment_light.jpg}
        \put (2,38) {Gaia ist der h\"ochste Elementargeist der Erde. Die Menschen stellen sich die}
        \put (2,34) {Personifikation der Erde als eine Frau mit leichtem, wei"sen Gewand vor.}
        \put (2,30) {Um ein Druide zu werden bedarf es einer jahrelangen Ausbildung, um}
        \put (2,26) {schlie"slich durch T\"atowierungen an Armen und Beinen an die Erde gebunden}
        \put (2,22)  {zu werden. Gaia wird ebenfalls als G\"ottin verehrt.}
        \put (2,18) {Die Elementargeister stellen das Grundger\"ust der Natur dar und k\"ummern sich}
        \put (2,14) {um die Bed\"urfnisse der Tier- und der Pflanzenwelt. Neben Gaia als h\"ochster}
        \put (2,10) {Elementargeist, sind die unteren Elementargeister folgenderma"sen aufgebaut:}
        \put (2,6)  {Elementargeister... der Kontinentalplatten $\rightarrow$ der \"Okozonen $\rightarrow$ der L\"ander $\rightarrow$}
        \put (2,2)  {der Biome und schlie"slich kleinere Elementargeister.}
    \end{overpic}
\end{figure}

\newpage

Nachdem seine Heimat von Orks \"uberfallen wurde, nahm sich Gaia seiner an und f\"uhrte ihn auf den Weg der Flora und Fauna. N\"amlich zu einem Druiden, der ihn ausbildete und an die Erde band. Auch wenn er auf andere Menschen als weltfremd gilt und er gro"se Siedlungen eher meidet, verbindet ihn der Hass auf die Orks mit den Soldaten des K\"onigs. Als Bewahrer der Natur sieht Thurmond sich in der Pflicht in eine unbekannte Welt aufzubrechen.

%%% TODO: zu veröffentlichen in der folgenden version nach der session am 19.02.2019
%\subsection{Lancelin Kisk}
%
%\textit{''Bei Hesinde! Ihr seid ein Schreiberling? Dann m\"usste man ja achtgeben was man tut und spricht, weil man ja ahnen muss, dass es die ganze Welt erf\"ahrt! - Bleibt mir also vom Leibe.''}
%
%\begin{figure}[h]
%    \begin{overpic}[width=\textwidth, height=0.15\textheight]{../img/parchment_light.jpg}
%        \put (2,18) {Hesinde ist die G\"ottin des Wissens, der Kunst und der Magie. Der}
%        \put (2,14) {'Immerw\"ahrende Hort der Hesindianischen Gaben', so der Name der Kirche,}
%        \put (2,10)  {versucht die gesamte Geschichte niederzuschreiben und Hesindes Gaben unter}
%        \put (2,6)  {den Menschen zu verteilen. In der Alchimie wird der Hesindemond dem}
%        \put (2,2)  {Element Kraft zugeordnet.}
%    \end{overpic}
%\end{figure}
%
%Ja, so k\"onnte ihm ein Abenteurer begegnen, schaute Lancelin just einmal von seinen B\"uchern empor. Lancelin ist ein B\"ucherwurm durch und durch. Sein langj\"ahriger Studienfreund und er schlugen sich schon die eine oder andere Nacht mit verborgenem Wissen, dunklen Geheimnissen und waghalsigen Abenteuern aus den Bibliotheken der Universit\"at um die Ohren. Einst erhielt er eine Transkriptionsarbeit von einem Magier, bei der er sein magisches Potential entdeckte und dies st\"andig auszubauen sucht. Der auch sprachlich begabte Student h\"atte damals nicht ansatzweise denken k\"onnen, dass er sich nun mitten in einem gro"sen Abenteuer wiederfinden w\"urden.

\newpage

%%%%%%%%%%%%%%%%%%%%%%%%
% Prequel              %
%%%%%%%%%%%%%%%%%%%%%%%%
\section{Prequel}

\lettrine{W}{ as} in den Wintermonaten mit dem K\"onigreich passiert, ist sagenhaft. Die sonst dichten W\"alder aus Nadel- und Laubgeh\"olz ragen aus dem Grund wie Z\"ahne und der Schnee, der sich vielerorts niederlegt, bedeckt weite Landstriche mit einer sanften, wei"sen Decke. In den n\"ordlichen Regionen peitsch ein kalter, rauher Wind, w\"ahrend sich im S\"uden die Tier- und Pflanzenwelt vom hei"sen Sommer erholt. Das Ende des Winters ist nun fast da und der Fr\"uhling h\"alt seinen Fu"s in die T\"ur.\\ \\
Ein halbes Jarhzehnt ist es her, dass K\"onig Thalamos, ''Zweitgeborener des Thorus, Herrscher des Reiches, Ehrerbieter der Zwerge und Diplomat der Zw\"olf'' (und neuerding, aber noch inoffiziell: 'Volksvereiniger'), den Thron nach dem Tod seines Bruders, der in der Schlacht bei Goldwasser gegen die Orks gefallen ist, bestieg. Seit diesem Tage befindet sich das Land im st\"andigen Wandel, da Thalamos beschlossen hatte, das Reich wieder in Baronien (also viele Unterreiche) aufzuteilen. Neue Positionen und Titel am k\"oniglichen Hofe werden vergeben und es entstehen somit die unterschiedlichsten Staatsgrenzen. Ein selbstorganisiertes und gutes Reich ist der sehnlichste Traum Thalamos', auch weil sein gro"ser Bruden im Begriff stand, das Verm\"achtnis ihres Vaters in den Untergang zu f\"uhren.\\ \\

\lettrine{F}{ erdork} ist schon immer ein belibtes Ziel f\"ur Reisende, die ein Abenteuer suchen, gewesen. Im Hafen herrscht ein reges Treiben, viele Schiffe legen an und wieder ab, jedermann geht seinem Tagewerk nach und vor allem ist das k\"onigliche Milit\"ar \"ublicherweise nur wenig vertreten. Was dem einen oder anderen gef\"allt, welcher das schnelle Geld sucht, aber dabei unentdeckt bleiben m\"ochte.

\begin{figure}[h]
    \begin{overpic}[width=\textwidth, height=0.20\textheight]{../img/parchment_light.jpg}
        \put (2,26)  {Die komplett ummauerte Stadt Ferdok an der Ferdok-M\"undung steht als}
        \put (2,22)  {Handelsstadt unter der Herrschaft einer neu enstandenen Baronie des K\"onigs.}
        \put (2,18)  {Die Lebensweise der Bev\"olkerung ist traditionell und obrigkeitsh\"orig. Der}
        \put (2,14)  {Seehafen ist gut besucht und beherbergt auch eine Vielzahl an Hausbooten}
        \put (2,10)  {und bewohnten Wracks. Im ber\"uhmten Loch des Gro"sen Platzes vor dem}
        \put (2,6)  {F\"urstenpalast kann man Gefangene von der Stra"se aus betrachten. Die Stadt}
        \put (2,2)  {wird von Reisfeldern und Sumpfland eingeschlossen.}
    \end{overpic}
\end{figure}

\newpage

Wer spannende Abenteuer bei einem guten Krog Met oder Bier h\"ohren m\"ochte, geht am besten ins Gasthaus 'Das Pony'. Diese Taverne ist auch der Ort in Ferdok, wo Ailan Abdan, der gro"se Seefehrer und Pirat, das erste Mal von einer mysteri\"osen Insel erz\"ahlt:\\

\lettrine{L}{ aute} Musik erschallt aus eine der hinteren Ecken, in die sich einige Musikanten gestellt haben, um in Hesindes Namen zu feiern (W\"are die Musik sehr harmonisch, k\"onnte man vermutlich behaupten sie spielten in Rahjas Namen). Dominant ist hier der Pfeifenbalg und einige Pfl\"oten aus Holz.

\begin{figure}[h]
    \begin{overpic}[width=\textwidth, height=0.15\textheight]{../img/parchment_light.jpg}
        \put (2,18)  {Rahja ist in der zw\"olfg\"ottlichen Mythologie die G\"ottin der Liebe, Sch\"onheit,}
        \put (2,14)  {sexuellen Freuden, des Rausches, des Weines und der Harmonie. Ihr ist der}
        \put (2,10)  {zw\"olfte Monat zugeordnet. Im K\"onigreich ist die Redewendung verbreitet,}
        \put (2,6)   {dass sich zwar viele G\"otter um das Wohl der Menschen k\"ummern, aber nur}
        \put (2,2)   {Rahja darum, was sie wirklich wollen.}
    \end{overpic}
\end{figure}

Die Theke erstreckt sich \"uber die schmale Seite des Schankraumes und rege Gespr\"ache zu Politik und Tratsch finden statt. Ein runder, gro"ser Tisch in der Mitte der Stube erregt mehr Aufmerksamkeit. Ailan Abdan sitzt von vielen Leuten umringt und erz\"ahlt eine seiner waghalsigen Geschichten. Doch heute scheint irgendetwas anders zu sein.\\

\textit{''Glaubt mir Leute, die Insel die ich mit meiner Crew gefunden habe ist der Wahnsinn! Durch St\"urme, Strudel und Seeungeheuer sind wir gesegelt und schlie"slich fanden wir sie.''\\
''Wo liegt die Insel?''\\
''Das ist wieder eine dieser erfundenenen Geschichten!''}\\
''Was f\"ur Ungeheuer?!''\\
- Viele Fragen und Spekulationen gehen wild durch den Raum und Ailan blickt nur mit gierigem Blick in die Runde -\\ \textit{''Wenn ihr mehr h\"oren wollt, m\"usst ihr mir schon noch ein paar Bierchen holen!''} lacht er und setzt zum Trinken an.\\

Pl\"otzlich springt die T\"ur zur Taverne auf und drei schwer gepanzerte Gardisten des K\"onigs betreten das Gasthaus. Zielstrebig gehen sie auf Ailan zu, der auch, als diese ihn mit seinem Namen ansprechen, nicht den Krug absetzt und gen\"usslich weitertrinkt.

\newpage

\textit{''Ich wiederhole mich nur ungern!''}\\
- Ein ernst dreinblickender Gardist schaut den Piraten durchdringend an -\\
\textit{''HAH! Ich wei"s genau was ihr wollt, ihr Maultierkl\"oten!}\\

DAS hatte hier noch niemand gewagt. Einem K\"onigsgardisten eine Beleidigung dreist ins Gesicht zu sagen, worauf immerhin eine harte Strafe stand. Das rege Treiben in der Taverne ist schlagartig dahin. Nach einigen Augenblicken zuckt die Hand des Mannes, der die Beleidigung empfangen hatte, zum G\"urtel, woraufhin Ailan den Griff seines Dolches umklammert, der sich an seinem G\"urtel befindet. Als der Gardist dies bemerkt, zieht er langsam seine Hand zur\"uck und mit ihr einen gesiegelten Brief. Die Lage beruhigt sich wieder ein wenig. Nachdem der Gardist den Brief vor Ailan auf den Tisch legt, raunt dieser mit grimmiger Miene:\\

\textit{''Irgendwas kriegen wir dich.''}\\
\textit{''Ihr k\"onnt mir nichts! Der K\"onig selbst hat wahrscheinlich sehr viel Interessen von meinem kleinen Abenteuer zu h\"oren, darum seid ihr doch hier!''} lacht der sichtlich erleichterte Pirat\\
\textit{''Und au"serdem seid ihr dazu viel zu dumm''}\\
\textit{''\"Uberspann den Bogen nicht!''} entgegnet sein Gegen\"uber und macht auf dem Absatz kehrt. Ein kurzer Wink zu seiner Gesellschaft und die Gepanzerten verlassen die Schankstube.\\

Alle Augen sind auf Ailan gerichtet. Dieser richtet sich langsam auf, streckt sich und verl\"asst die Taverne ohne den Brief an sich zu nehmen mit dem Gewissen, dass niemand es wagen w\"urde einen gesiegelten Brief des K\"onigs zu \"offnen. Der Tisch mit dem Brief wird f\"ur den restlichen Tag nicht mehr besetzt und sogar gemieden werden.\\

 \newpage

\lettrine{E}{ inige} Wochen gingen ins Land und es ist allgemein bekannt, dass K\"onig Thalamos den Piraten Ailan Abdan mit der \"Uberfahrt einiger Siedler, Abenteurer und Soldaten betraut. Was f\"ur den Piraten dabei herausspringt, ist nicht bekannt. Aufgabe ist die Landerschlie"sung eines neuen Eilands, welches der Pirat vor wenigen Monaten entdeckte. Der Herrscher des Reiches sendet deswegen Botschaften in das ganze K\"onigreich, um mutige Siedler und Recken auszuw\"ahlen, die die Fahrt antreten sollen. Auch Schurken aus den Kerkern der Hauptstadt sollen die Fahrt antreten, aber nur diejenigen, die sich einen Neuanfang erarbeiteten.\\

So kommt es, dass die Recken, um die sich diese Geschichte hier dreht, zueinander finden. Heagan, Kerlia, Koriel, Rurik und Thurmond. % TODO: Lancelin hinzufuegen nach der Session am 19.02.2019
Ein jeder ist im Besitzt einer Aufforderung der \"Uberfahrt beizuwohnen. An dieser Stelle ist zu vermerken, dass einer Widersetzung dieser 'Aufforderung' eine erzwungene Kooperation mit Waffengewalt folgt.

\section{Kapitel I: Eine sonderbarer Brief}

\end{document}