%% Author: Adrian Helberg
%% 05.02.2019

\documentclass[11pt, openany]{article} % Preamble
\usepackage{amsmath} % Packages
\usepackage[english, ngerman]{babel}% Language
\usepackage{fontspec} % Font
\usepackage[urf8]{inputenc}
\setmainfont[ItalicFont=Dominican Italic]{Dominican}
\usepackage[absolute]{textpos} % Positioning

% Title
\title{\fontsize{40pt}{42pt}\selectfont Dungeons \& Dragons \\
Rathians Hort
\author{Blinzlar}
\date{Letzte Federf\"uhrung: \today}
\begin{textblock}{4}(6,10)
\Huge Reisetagebuch
\end{textblock}
}

% Table of contents
\addto\captionsngerman{
\renewcommand{\contentsname}
{Schlachtplan}
}

% Content
\begin{document}
\maketitle
\newpage
\tableofcontents
\newpage

%%%%%%%%%%%%%%
% Einleitung %
%%%%%%%%%%%%%%
\section{Einleitung}

Gegr\"u"st seist du werter Wanderer, abenteuerlicher Gesell!\\ Das Fabelhafte, das du hier lesen wirst, schreibt sich als Erfahrung einer Gruppe von wackeren Recken nieder, die einmal Abenteuerluft geschnuppert, auf eine Reise ins Unbekannte aufbrachen. Nimm nur all deinen Mut zusammen und folge ihnen durch Raum und Zeit und werde Zeuge einer Menge Spa"s, Frust und Angstschwei"s! Man F\"uhre sich folgende Zitate zu Gem\"ute:

\begin{quote}
    \textit{"Die Geschichten des Reisetagebuchs entspringen einer Reihe von D\&D-Sessions, sprich: Eine Aneinanderreihung von Bl\"odeleien, Unsinn und empirischer Feldforschung"}
    \begin{flushright}
        - Ein D\&D-Spieler
    \end{flushright}
\end{quote}

\begin{quote}
    \textit{"Wer im Gasthaus sitzt, sollte nicht mit Schweinen werfen!"}
    \begin{flushright}
        - Ein Zwerg
    \end{flushright}
\end{quote}

\begin{quote}
    \textit{Elfische Volksweisheit, stark vereinfacht: "Wer seine Artgenossen befriedigt, hat Selbstbefriedigung erlangt"}
    \begin{flushright}
        - Eine Elfe
    \end{flushright}
\end{quote}

Zwergenfaustregel f\"ur diese Abschrift:
\begin{enumerate}
    \item[1.] Ja, es sind Tiere beim Verfassen dieses Buchs oder w\"ahrend des D\&D-Spiels zu Schaden gekommen. Wir hatten halt Hunger!
    \item[2.] Der Meister (das bin ich) steht \"uber allem. MUHAHA!
    \item[3.] \"Ahnlichkeiten mit den Figuren im Buch gegen\"uber der Wirklichkeit sind mir egal
    \item[4.] Weitere folgen
\end{enumerate}

%%%%%%%%%%%%%%%%%%%%%%%%
% Charaktervorstellung %
%%%%%%%%%%%%%%%%%%%%%%%%
\section{Charaktervorstellung}

\subsection{Heagan}
\subsection{Kerlia}
\subsection{Koriel}
\subsection{Rurik}
\subsection{Thurmond}
\subsection{Turian}

\end{document}